
\documentclass[11pt]{article}

\usepackage[margin=1in]{geometry}
\usepackage{amsmath, amssymb}
\usepackage{graphicx}
\usepackage{booktabs}
\usepackage{siunitx}
\usepackage{hyperref}
\usepackage{microtype}

\title{A Reproducible SPARC Rotation-Curve Fitting Pipeline and SIDM Toolkit (sidmkit)}
\author{SIDMKit Collaboration}
\date{January 7, 2026}

\begin{document}
\maketitle

\begin{abstract}
We present \texttt{sidmkit}, a lightweight Python toolkit for self-interacting dark matter (SIDM) phenomenology, together with a submission-focused pipeline for batch fitting SPARC rotation curves from the published \texttt{rotmod} files. 
The rotation-curve pipeline fits two standard halo profiles (NFW and Burkert) plus stellar mass-to-light parameters using a robust, chunkable batch interface designed for large samples. 
We emphasize that the SPARC analysis reported here is \emph{phenomenological} (halo-profile baseline fitting) and does not yet constitute a full microphysical SIDM inference of galaxy cores. 
All outputs (per-galaxy plots, merged summaries, population diagnostics) are produced by a reproducible command-line workflow and are intended as a reliable foundation for subsequent microphysics-to-halo mapping and more detailed astrophysical systematics modeling.
\end{abstract}

\section{Scope and philosophy}
Small-scale tensions in collisionless cold dark matter motivate studying SIDM models, but analysis pipelines often couple multiple layers at once: microphysics (particle model), kinetic theory (cross sections and velocity averages), halo modeling (core formation), and data likelihoods. 
\texttt{sidmkit} separates these concerns on purpose. 
This submission package contains:
(i) a general SIDM phenomenology library (cross sections, velocity averaging, constraints/likelihood scaffolding), and
(ii) a \emph{standalone} SPARC rotation-curve batch fitter that provides a transparent baseline comparison between cuspy (NFW) and cored (Burkert) halo profiles using the SPARC \texttt{rotmod} decomposition.

\paragraph{Critical note.}
The SPARC pipeline here is a \emph{baseline}: it tests whether a cored or cuspy parametric halo better matches the published rotation curves under simple assumptions. 
It does \textbf{not} claim that Burkert halos are ``the truth'' nor that SIDM microphysics has been inferred from these fits. 
Important systematics (distance and inclination uncertainties, non-circular motions, beam smearing, baryonic modeling errors, and correlated uncertainties) are not modeled in this baseline.



\section{SIDM microphysics modules in \texttt{sidmkit}}
The rotation-curve fitter in this submission package is intended to be a stable, transparent \emph{baseline}. 
In parallel, \texttt{sidmkit} provides a modular SIDM phenomenology layer that can be used to connect particle models to astrophysical observables.

\subsection{Yukawa interactions and transfer cross sections}
A commonly studied SIDM scenario is elastic scattering in an attractive (or repulsive) Yukawa potential,
\begin{equation}
V(r) = \pm \frac{\alpha_\chi}{r} e^{-m_\phi r},
\end{equation}
where $\alpha_\chi$ is a dark-sector coupling and $m_\phi$ is the mediator mass.
From the differential cross section $d\sigma/d\Omega$, two angle-weighted quantities are especially useful for structure formation:
the momentum-transfer cross section and the viscosity cross section,
\begin{align}
\sigma_T(v) &= \int d\Omega\, (1-\cos\theta)\, \frac{d\sigma}{d\Omega},\\
\sigma_V(v) &= \int d\Omega\, (1-\cos^2\theta)\, \frac{d\sigma}{d\Omega}.
\end{align}
These suppress forward scattering, which is inefficient at isotropizing particle orbits.
The toolkit implements a practical hybrid strategy:
Born-limit expressions where applicable, classical-regime approximations at large coupling/low de~Broglie wavelength,
and (optionally) a partial-wave phase-shift solver in intermediate regimes.
This layered approach is common in the SIDM literature; it is computationally efficient but must be treated as an approximation whose regime boundaries should be stress-tested for any specific application.

\subsection{Velocity averaging}
Astrophysical constraints are typically quoted at characteristic relative velocities (e.g.\ $\sim\!30$ km/s for dwarf galaxies and $\sim\!1000$ km/s for clusters).
When a velocity distribution $f(v)$ is needed, \texttt{sidmkit} provides utilities to compute velocity averages such as
\begin{equation}
\langle \sigma_T v \rangle = \int_0^\infty dv\, f(v)\, \sigma_T(v)\, v,
\end{equation}
with Maxwell--Boltzmann distributions as a common default.
This step matters whenever $\sigma(v)$ is strongly velocity dependent.

\subsection{Constraint evaluation as a convenience layer}
The package also includes a small collection of literature-derived constraint points for $\sigma/m$ at representative velocities.
These are \emph{not} a substitute for a full re-analysis; rather they provide a consistent, scriptable way to sanity-check a parameter point before investing in expensive inference.
The correct scientific posture is to treat these as orientation tools and to consult the original sources for definitive statements.


\section{SPARC pipeline methods}

\subsection{SPARC \texttt{rotmod} inputs}
Each SPARC \texttt{*\_rotmod.dat} file provides, as a function of radius $r$ (kpc), an observed rotation speed $V_{\rm obs}$ (km/s) and uncertainty $\sigma_V$, along with template velocity contributions for gas, stellar disk, and (when present) stellar bulge.
The model prediction combines these components in quadrature:
\begin{equation}
V_{\rm model}^2(r) = V_{\rm gas}^2(r) + \Upsilon_{\star,d}\,V_{\rm disk}^2(r) + \Upsilon_{\star,b}\,V_{\rm bul}^2(r) + V_{\rm DM}^2(r),
\end{equation}
where $\Upsilon_{\star,d}$ and $\Upsilon_{\star,b}$ are disk and bulge stellar mass-to-light parameters.

\subsection{Halo models}
We fit two standard parametric halos as a baseline.

\paragraph{NFW.}
The Navarro--Frenk--White density profile is
\begin{equation}
\rho_{\rm NFW}(r) = \frac{\rho_s}{(r/r_s)(1+r/r_s)^2}.
\end{equation}
The enclosed mass $M(r)$ has a closed form, yielding $V_{\rm DM}^2(r) = G M(r)/r$.

\paragraph{Burkert.}
The Burkert profile is a common cored phenomenological model:
\begin{equation}
\rho_{\rm Bur}(r) = \frac{\rho_0 r_0^3}{(r+r_0)(r^2+r_0^2)},
\end{equation}
with an analytic enclosed mass and circular speed.

\subsection{Objective function and priors}
For each galaxy we minimize a weighted residual:
\begin{equation}
\chi^2 = \sum_i \left(\frac{V_{\rm model}(r_i)-V_{\rm obs}(r_i)}{\sigma_V(r_i)}\right)^2.
\end{equation}
For numerical stability in batch mode, the default fits include weak Gaussian priors on $\Upsilon_\star$ (configurable); this yields a maximum-a-posteriori (MAP) estimate.
For strict likelihood-based information criteria comparisons, the pipeline supports \texttt{--no-priors} (pure maximum likelihood).

\subsection{Model comparison diagnostics}
We report reduced chi-square $\chi^2_\nu = \chi^2/\nu$ with $\nu = N-k$ degrees of freedom.
For each galaxy we also compute AIC/BIC from the data-only $\chi^2$:
\begin{align}
{\rm AIC} &= \chi^2 + 2k,\\
{\rm BIC} &= \chi^2 + k\ln N.
\end{align}
Since both halo models have the same parameter count $k$ in this baseline setup, $\Delta{\rm BIC}$ is equivalent (up to an additive constant) to $\Delta\chi^2$ for per-galaxy model preference.

\subsection{Batch execution: chunking for submission-grade runs}
Full-sample plotting is slow if run monolithically. The pipeline is designed to be chunked:
\begin{itemize}
\item \texttt{--skip N}: skip the first $N$ galaxies in sorted order,
\item \texttt{--limit M}: process at most $M$ galaxies after the skip,
\item \texttt{--resume}: avoid recomputing galaxies that already have outputs.
\end{itemize}
This enables sequential chunking or parallel execution across multiple processes/nodes.

\section{Validation and robustness checks}
We include:
(i) unit tests for parsing and fitting on synthetic rotmod-like inputs,
(ii) deterministic JSON/CSV summaries per chunk, and
(iii) population-level plots that flag common failure modes (e.g.\ parameter-bound saturation).
We report the fraction of NFW fits saturating the upper bound on $\log_{10}(r_s/{\rm kpc})$ as a practical indicator of poorly constrained NFW scale radii in the baseline parameterization.

\section{Results on the full SPARC rotmod sample}

Using the attached run outputs (191 galaxies, two halo models each), we obtain:
\begin{itemize}
\item $N=191$ galaxies with both NFW and Burkert fits.
\item Median $\Delta{\rm BIC} = 1.812$, mean $\Delta{\rm BIC} = 12.915$, where $\Delta{\rm BIC} = {\rm BIC}_{\rm NFW}-{\rm BIC}_{\rm Burkert}$ (positive values favor Burkert).
\item Fraction preferring Burkert ($\Delta{\rm BIC}>0$): 0.654.
\item ``Strong'' Burkert preference fraction ($\Delta{\rm BIC}>6$): 0.325; strong NFW preference fraction ($\Delta{\rm BIC}<-6$): 0.147.
\item Median reduced chi-square: NFW 1.251, Burkert 0.708.
\item NFW scale-radius fits saturate the upper bound in 0.215 of galaxies (strict $10^{-9}$ tolerance), indicating that a non-negligible subset of NFW fits is effectively ``pushing'' toward very large $r_s$ in this baseline parameterization.
\end{itemize}

\begin{figure}[t]
  \centering
  \includegraphics[width=0.78\linewidth]{figures/delta_bic_hist.png}
  \caption{Distribution of $\Delta{\rm BIC} = {\rm BIC}_{\rm NFW}-{\rm BIC}_{\rm Burkert}$ across the sample. Positive values favor Burkert. Because both models have the same number of fitted parameters, $\Delta{\rm BIC}$ is essentially a re-scaled $\Delta\chi^2$ in this baseline comparison.}
  \label{fig:delta_bic}
\end{figure}

\begin{figure}[t]
  \centering
  \includegraphics[width=0.62\linewidth]{figures/chi2red_scatter.png}
  \caption{Per-galaxy reduced chi-square comparison. Points below the diagonal indicate galaxies where Burkert yields a smaller $\chi^2_\nu$ than NFW.}
  \label{fig:chi2_scatter}
\end{figure}

\begin{figure}[t]
  \centering
  \includegraphics[width=0.78\linewidth]{figures/IC2574_fit.png}\\[0.5em]
  \includegraphics[width=0.78\linewidth]{figures/NGC2403_fit.png}\\[0.5em]
  \includegraphics[width=0.78\linewidth]{figures/NGC2685_fit.png}
  \caption{Example galaxy fits with residuals. Each panel shows observed rotation speeds, baryonic components, total baryons, and best-fit NFW/Burkert totals. These examples illustrate that preference can vary by object; the population-level metrics aggregate over this heterogeneity.}
  \label{fig:examples}
\end{figure}

\section{Limitations and next steps}
This baseline should be treated as a \emph{starting point}.
Key limitations:
\begin{itemize}
\item The fits use published SPARC templates and per-point uncertainties without modeling covariance or additional observational systematics.
\item The NFW and Burkert parameterizations here are phenomenological; physical priors (e.g.\ $M_{200}$ and concentration) could reduce degeneracies and bound-saturation.
\item Microphysical SIDM parameters (e.g.\ Yukawa mediator models) are not yet mapped to halo core sizes in this SPARC pipeline. That mapping requires additional assumptions and calibration and is best implemented as a distinct inference layer building on the stable batch-fitting substrate presented here.
\end{itemize}

\section{Reproducibility}
All commands used to generate the attached outputs are included in the submission runbook. The key workflow is:
\begin{enumerate}
\item chunked fits: \texttt{python -m sidmkit.sparc\_batch batch --skip ... --limit ...}
\item merge summaries: \texttt{python -m sidmkit.sparc\_batch merge ...}
\item population report: \texttt{python -m sidmkit.sparc\_batch report ...}
\end{enumerate}

\begin{thebibliography}{99}
\bibitem{SPARC} F. Lelli, S. McGaugh, and J. Schombert, ``SPARC: Mass Models for 175 Disk Galaxies with Spitzer Photometry and Accurate Rotation Curves,'' \emph{Astronomical Journal} 152 (2016) 157.
\bibitem{NFW} J. Navarro, C. Frenk, and S. White, ``A Universal Density Profile from Hierarchical Clustering,'' \emph{Astrophysical Journal} 490 (1997) 493.
\bibitem{Burkert} A. Burkert, ``The structure of dark matter halos in dwarf galaxies,'' \emph{Astrophysical Journal Letters} 447 (1995) L25.
\bibitem{TulinYu} S. Tulin and H.-B. Yu, ``Dark matter self-interactions and small scale structure,'' \emph{Physics Reports} 730 (2018) 1--57.

\end{thebibliography}

\end{document}
